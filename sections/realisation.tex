\newpage

\titleformat % design des titres des chapitres
{\chapter}
[display]
{\centering\normalfont\Large\scshape\bfseries}
{\rule[3pt]{0.15\linewidth}{3pt}\quad\chaptertitlename~\thechapter\quad \rule[3pt] {0.15\linewidth}{3pt}}
{0\baselineskip}%espace vertical entre chapitre et nom du chapitre
{\rule{\linewidth}{0.5pt}\break\Huge}
[\vspace{-0.5\baselineskip}\rule{\linewidth}{0.5pt}\vspace{0\baselineskip}]

\let\clearpage\relax% Stop LaTeX from going to a new page; and
\vspace*{5.5cm}%

\chapter{Réalisation du projet}
Dans ce chapitre, je vais présenter la réalisation du projet. D’abord je vais exposer
les interfaces de l’application, puis nous décrirons les différentes fonctionnalités de ces dernières.

\newpage

\section{Interface d'acceuil}

\begin{figure}[!h]
    \centering
    \includegraphics[scale=0.4,keepaspectratio]{Rapport de stage PFE chez DXC/figures/Home_Page.png}
    \caption{Interface d'accueil}
\end{figure}

La page d'accueil regroupe les différentes procédures possibles que propose l'application dans un menu verticale à savoir les procédures en relation avec les comptes, le service line, l'interlock de ces deux dernier mais aussi la gestion des demandes.
\\[0.1cm]

Commençons par donner une explication générale à chacune des sections présente dans le menu de la page d’accueil :
\\

\begin{itemize}
    %% ======================== Account Processes ===============================
    \item[\ding{118}] \textbf{Account Processes:}
    
        \vspace{0.5cm}
        %% ====================== Add demand to account ==============================
        \begin{itemize}
            \item[\textbullet] \textbf{Add demand to account:} Cette option permet à un administrateur de compte de faire une demande d'ajout de masse salariale à son compte, les demande d'ajout sont divisé en quatre parties :
            
            \begin{itemize}
                \item[\ding{51}] \textbf{Existing Account:} est utiliser pour ajouter une demande pour un client existant.
                \item[\ding{51}] \textbf{New Logo:} est utiliser pour ajouter une demande pour un futur client.
                \item[\ding{51}] \textbf{Roll-Off/Roll-On:} est utiliser lorsqu’un employé change de projet mais reste avec le même  compte. 
                \item[\ding{51}] \textbf{Placeholder - Add demand:} joue le rôle d'une entrée temporaire pour un futur besoin, cette dernière peut être convertit en une entrée VA dans le futur.
            \end{itemize} 
            
            %% ====================== Modification of demand ==============================
            \newpage
            \item[\textbullet] \textbf{Modification of demand:} Cette option permet a un administrateur de compte de modifier les information d'une position ou bien employé: 
            
            \begin{itemize}
                \item[\ding{51}] \textbf{Modification of demand:} qui est composé de deux lever a savoir Run-Off qui represente un employé dont le contrat arrive à son terme ou bien extension qui va permettre a l'administrateur du compte d'étendre sont contrat.
                \item[\ding{51}] \textbf{Placeholder Modification:} of demand qui est similaire au placeholder add demand expliqué precedement, cette dernière permet de préparer une entrée de modification pour un futur besoin, elle aussi peut être convertit en une entrée VA dans le futur.
            \end{itemize}
            
            \vspace{0.5cm}
            %% ====================== Optimize demand ==============================
            \item[\textbullet] \textbf{Optimize demand:} Cette option permet à un administrateur de compte comme son nom l'indique d'optimiser sa demande en effectuant une opération de work migration ou bien Productivity :
            
            \begin{itemize}
                \item[\ding{51}] \textbf{Work Migration/LPI:} est utilisé lorsequ'on veut migrer un employé vers un autre centre de delivery DXC.
                \item[\ding{51}] \textbf{Productivity:} Est utilisé lors d'un gain au niveau du coup de l'employé, par exemple en cas ou le remplacement d'une position nous coute moins chère.
            \end{itemize}
            
            \vspace{0.5cm}
            %% ====================== View submissions ==============================
            \item[\textbullet] \textbf{View submission :} Cette option permet à un administrateur de compte de voir ces entrées, de les modifier, d'annuler une entré, de créer à partir d'une entrée VA une entrée DCT mais aussi de dupliquer son entrée.
            
        \end{itemize}
        
    \vspace{0.5cm}
    
    %% ======================== Service Line Processes ===============================
    
    \item[\ding{118}] \textbf{Service Line Processes} 
        
        \vspace{0.5cm}
        
        \begin{itemize}
        
            %% ======================== SL Employees ===============================
            \item[\textbullet] \textbf{Employees:} Cette option permet au responsable du service line d'effectuer différente demande sur les employés à savoir : 
            \begin{itemize}
                \item[\ding{51}] \textbf{Modification of demand:} De la même manière qu'un administrateur de compte un responsable de service line peut effectuer une opération de run-off ou bien extension sur l'un des employés.
                \item[\ding{51}] \textbf{Work migration/LPI:} Il peut aussi effectuer une action de migration sur l'un des employés.
                \item[\ding{51}] \textbf{Productivity} Il peut également réaliser une modification de type productivity sur l'un des employés.
                \item[\ding{51}] \textbf{View submission :} Il dispose aussi d'une interface dans laquelle il peut parcourir ces différentes entrées, les modifier ou bien dupliquer.
            \end{itemize} 
            
            \vspace{0.5cm}
            
            %% ======================== Subcontractors ===============================
            
            \item[\textbullet] \textbf{Subcontractors :} Cette option est resérvé au subcontractors, les subcontractors sont les centre de delivery qui ne sont pas a 100\% DXC, Par exemple DXC technology Maroc est une jointure entre la CDG (49\%) et DXC technoloty (51\%) ce qui fait d'elle un subcontractor.
            Pour l'instant le design des differentes opération de cette option est toujours en discussion.
            
        \end{itemize}
        
   
    \newpage
    %% =========================== Interlock Process =============================
    \item[\ding{118}] \textbf{Interlock Process}
    
    \vspace{0.5cm}
    
        \begin{itemize}
        
            %% ======================== Interlock from Account ===============================
            \item[\textbullet] \textbf{Interlock Submission from Account:} Cette option permet au responsable de service line de confirmer l’information saisit par les responsables du compte mais aussi ajouter d'autre information et ensuite approuver la soumission.
            
            \vspace{0.5cm}
            
            %% ======================== Interlock from SL ===============================
            \item[\textbullet] \textbf{Interlock Submission from Service Line:} Cette option permet au responsable des compte de confirmer l’information saisit par les responsables du service line mais aussi ajouter d'autre information et ensuite approuver la soumission.
            
            \vspace{0.5cm}
            
            %% ======================== Interlock from Finance ===============================
            
            \item[\textbullet] \textbf{Interlock Submission from Service Line:} Cette option permet au responsable de la finance de confirmer eux aussi les entrées de ces deux derniers, Pour l'instant le processus est toujours en cours de discussion.

        \end{itemize}
    
    %% =========================== Demand Management =============================
    
    \vspace{0.5cm}
    
    \item[\ding{118}] \textbf{Demand Management}
    
    \vspace{0.5cm}
    
        \begin{itemize}
            
            %% ====================== Position Level Management ========================
            \item[\textbullet] \textbf{Position level management:} Cette interface est consacrée à la saisie de demande par les comptes dans un format DCT.
                
                \begin{itemize}
                    \item[\ding{51}] \textbf{Create demand:} Pour créer une nouvelle demande.
                    \item[\ding{51}] \textbf{Update demand:} Pour modifier une demande existante.
                    \item[\ding{51}] \textbf{View submission} Une interface qui permet à l'utilisateur de voir ces différentes soumissions de les modifier, de consulter les commentaires de la part de l'examinateur de la demande, d'attacher un fichier pièce jointe d'approbation, de dupliquer une demande, mais aussi de l'annuler.
                \end{itemize}
                
            \vspace{0.5cm}
            
            %% =========================== Project  Level Management ==============================
            
            \item[\textbullet] \textbf{Project level management:} Cette interface est consacrée à la saisie de nouveau project.
            
                \begin{itemize}
                        \item[\ding{51}] \textbf{Create a new project:} Pour créer un nouveau projet.
                        \item[\ding{51}] \textbf{Update an existing project:} Pour modifier un projet existant.
                        \item[\ding{51}] \textbf{Close an existing project:} Pour fermer un projet existant.
                        \item[\ding{51}] \textbf{View submission} Cette interface permet à l'utilisateur de voir ces différentes soumissions.
                \end{itemize}
            
            %% ================================ Demand team ====================================
            
            \vspace{0.5cm}
            
            \item[\textbullet] \textbf{Demand team :} Cette interface est consacrée aux experts qui vont examiner les différentes demandes pour les valider.
            
             \begin{itemize}
                        \item[\ding{51}] \textbf{Demand validation check:} Cette interface contient la liste de toutes les entrées soumises par les comptes, l'examinateur en cliquant sur une entrée peut l'assigner à lui-même et ensuite effectuer les différentes vérifications, par la suite il peut la valider.
                        \item[\ding{51}] \textbf{DCT position export:} Après avoir valider une soumission DCT l'examinateur peut ensuite accéder à cette interface qui va lui permettre d'exporter toutes les soumissions validées pour les ajouter à l'outil corporel de DXC qui gère toutes les demandes.
                \end{itemize}
            
        \end{itemize}
    
\end{itemize}

La page d'accueil contient aussi un aperçu rapide du nombre de compte affecté pour l'utilisateur, le nombre d'entré saisie par l'utilisateur.
\\[0.3cm]
On peut aussi être renvoyer vers une autre application qui nous permet d'enregistrer les différentes suggestions des utilisateurs ou bien envoyer un email en cas de problème à la mailbox de support.

%Add demand to account
\section{Compte - Ajout de masse salariale}


% ========================= Existing account =================================
\subsection{Compte existant}

\subsubsection{Présentation de l'interface :}

L'interface suivante permet a un responsable de compte de faire une demande d'ajout, c'est a dire ajouter une position pour une opportunité ou bien projet.


\subsubsection{Regle de gestion:}

Le tableau suivant represente les different regle de gestion que l'interface doit respecter :

\vspace{0.2cm}

\tikzset{ 
    table/.style={
        matrix of nodes,
        row sep=-\pgflinewidth,
        column sep=-\pgflinewidth,
        nodes={
            rectangle,
            draw=black
        },
        minimum height=1.5em,
        text depth=0.5ex,
        text height=2ex,
        nodes in empty cells,
%%
        every even row/.style={
            nodes={fill=gray!70}
        },
        column 1/.style={
            nodes={text width=10em,align=center}
        },
        column 2/.style={
            nodes={text width=24em}
        },
        column 3/.style={
            nodes={text width=5em,align=center}
        },
        row 1/.style={
            nodes={
                font=\bfseries,
                align=center,
                fill=black,
                text=white,
            }
        },
        row 2/.style={
            nodes={
                fill=black,
                text=white,
                text depth=4ex,
            }
        },
        row 3/.style={
            nodes={
                fill=white,
                text=black,
                text depth=4ex,
            }
        },
        row 4/.style={
            nodes={
                fill=black,
                text=white,
                text depth=4ex,
            }
        },
        row 5/.style={
            nodes={
                fill=white,
                text=black,
                text depth=4ex,
            }
        },
        row 6/.style={
            nodes={
                fill=black,
                text=white,,
                text depth=11ex,
            }
        },
        row 7/.style={
            nodes={
                fill=white,
                text=black,
                text depth=7ex,
            }
        }
    }
}

\begin{tikzpicture}

\matrix (first) [table,text width=6em]
{
 Regle de gestion  & Description & Type \\
 RG01 & Toutes les informations de l’étape 2 doivent être remplis & Métier \\
 RG02 & Si une information de l'étape 2 ou bien 1 est vide le bouton de soumission doit être désactiver. & Métier \\
 RG03 & La valeur du FTE doit etre entre 0 et 1 & Métier \\
 RG04  & Le champ monthly cost n'accepte que des chiffres & Métier \\
 RG05  & Si l'utilsateur selectionne "Yes" dans le champ "in account forcast", Il doit imperativement remplire les deux champ suivant a savoir le l'année fiscale et le mois. & Métier \\
 RG06  & Une fois l'entrée soumis tout les champs doivent etre réinitialiser puis envoyer l'utilisateur a l'interface "view submission" & IHM\\
};


\end{tikzpicture}


\newpage

\subsubsection{Fonctionnement de l'interface :}

La figure ci-dessous représente l'interface qui permet de faire une demande de croissance par un administrateur de compte :

\begin{figure}[!h]
    \centering
    \includegraphics[scale=0.4,keepaspectratio]{Rapport de stage PFE chez DXC/figures/existin_account.png}
    \caption{Ajout de demande - Compte existant}
\end{figure}

La procédure pour saisir une entrée de type croissance est le suivant :

\begin{enumerate}
    
    \item L'utilisateur commence par sélectionner le sub-lever, qui représente la catégorie de la demande d'ajout pour un projet ou bien une croissance stratégique ou autres ensuite il doit cliquer sur le bouton ok.
    \vspace{0.1cm}
    \item Ensuite l'utilisateur doit sélectionner le compte et le contrat qu'il souhaite.
    \vspace{0.1cm}
    \item Puis apparait diffèrent champ qu'il doit remplir, pour l'étape 1 il suffit de sélectionner un OPX-ID qui représente un identificateur d'une opportunité ensuite toutes les informations sont automatiquement remplis.
    \vspace{0.1cm}
    \item La deuxième étape est saisie par l'utilisateur il doit remplir toutes les informations de la demande
    \vspace{0.1cm}
    \item Finalement l'utilisateur a deux choix soit soumettre directement ou bien la soumettre en tant que brouillon s'il n'est pas sur ou souhaite changer les informations de l'entrée dans le futur

\end{enumerate}

\newpage

% =========================== New logo =================================

\subsection{New Logo}

\subsubsection{Présentation de l'interface :}

L'interface suivante permet à un responsable de compte de faire une demande d'ajout de la même manière que la précédente la seule différence est l'utilisation du contrat Salesforce a la place du nom du compte.

\subsubsection{Regle de gestion:}

L'interface doit respecter le même tableau de règle de gestion vu précédemment.

\subsubsection{Fonctionnement de l'interface :}

\begin{figure}[H]
    \centering
    \includegraphics[scale=0.4,keepaspectratio]{Rapport de stage PFE chez DXC/figures/New_Logo.png}
    \caption{Ajout de demande - New Logo}
\end{figure}

Ce formulaire fonctionne de la même façon que le précèdent la seule différence est dans le choix du compte, dans ce dernier le choix du compte est fait à travers le contrat Salesforce car le compte n'est pas encore dans la base de données financière de DXC d'où la nomination "New Logo".

\newpage

% =========================== Roll-Off / Roll-On =================================

\subsection{Roll-Off / Roll-On}

\subsubsection{Présentation de l'interface :}

L’interface suivante permet à un responsable de compte de faire une demande d’ajout, la différence est que dans ce cas un employé va changer de projet mais va rester avec le même compte.

\subsubsection{Regle de gestion:}

L’interface doit respecter le même tableau de règle de gestion vu précédemment.

\subsubsection{Fonctionnement de l'interface :}

\begin{figure}[H]
    \centering
    \includegraphics[scale=0.4,keepaspectratio]{Rapport de stage PFE chez DXC/figures/rolloff_rollon.png}
    \caption{Ajout de demande - Roll-Off / Roll-On}
\end{figure}

La procédure pour saisir une entrée de type Roll-Off/Roll-On est le suivant :

\begin{enumerate}
    
    \item L'utilisateur doit sélectionner le compte et le contrat qu'il souhaite.
    \vspace{0.1cm}
    \item L'étape 1 est divisé en deux, La première partie est utilisé pour collecter les informations en relation avec la nouvelle opportunité, la deuxième partie est utilisée pour assembler les informations en relation avec l'employé.
    \vspace{0.1cm}
    \item La deuxième étape est saisie par l'utilisateur il doit remplir toutes les informations pour compléter l'entrée.
    \vspace{0.1cm}
    \item Finalement l'utilisateur a deux choix soit soumettre directement ou bien la soumettre en tant que brouillon s'il n'est pas sur ou souhaite changer les informations de l'entrée dans le futur

\end{enumerate}

% =========================== PlaceHolder - Add Demand =================================

\subsection{PlaceHolder - Add Demand}

\subsubsection{Présentation de l'interface :}

L’interface suivante permet à un responsable de compte de créer une sorte de demande temporaire, c'est à dire ici l'utilisateur peut créer une demande de croissance avec un FTE supérieur à 1, ensuite la convertir en plusieurs entrée VA.

\subsubsection{Regle de gestion:}

Le tableau suivant represente les different regle de gestion que l'interface doit respecter :

\vspace{0.2cm}

\tikzset{ 
    table/.style={
        matrix of nodes,
        row sep=-\pgflinewidth,
        column sep=-\pgflinewidth,
        nodes={
            rectangle,
            draw=black
        },
        minimum height=1.5em,
        text depth=0.5ex,
        text height=2ex,
        nodes in empty cells,
%%
        every even row/.style={
            nodes={fill=gray!70}
        },
        column 1/.style={
            nodes={text width=10em,align=center}
        },
        column 2/.style={
            nodes={text width=24em}
        },
        column 3/.style={
            nodes={text width=5em,align=center}
        },
        row 1/.style={
            nodes={
                font=\bfseries,
                align=center,
                fill=black,
                text=white,
            }
        },
        row 2/.style={
            nodes={
                fill=black,
                text=white,
                text depth=4ex,
            }
        },
        row 3/.style={
            nodes={
                fill=white,
                text=black,
                text depth=4ex,
            }
        },
        row 4/.style={
            nodes={
                fill=black,
                text=white,
                text depth=4ex,
            }
        },
        row 5/.style={
            nodes={
                fill=white,
                text=black,
                text depth=4ex,
            }
        },
        row 6/.style={
            nodes={
                fill=black,
                text=white,,
                text depth=11ex,
            }
        },
        row 7/.style={
            nodes={
                fill=white,
                text=black,
                text depth=7ex,
            }
        }
    }
}

\begin{tikzpicture}

\matrix (first) [table,text width=6em]
{
 Regle de gestion  & Description & Type \\
 RG01 & Toutes les informations de l’étape 2 doivent être remplis & Métier \\
 RG02 & Si une information de l'étape 2 ou bien 1 est vide le bouton de soumission doit être désactiver. & Métier \\
 RG03 & La valeur du FTE peut etre superieur a 1 & Métier \\
 RG04  & Le champ monthly cost n'accepte que des chiffres & Métier \\
 RG06  & Une fois l'entrée soumis tout les champs doivent etre réinitialiser puis envoyer l'utilisateur a l'interface "view submission" & IHM\\
};


\end{tikzpicture}

\subsubsection{Fonctionnement de l'interface :}

\begin{figure}[H]
    \centering
    \includegraphics[scale=0.4,keepaspectratio]{Rapport de stage PFE chez DXC/figures/Placeholder_add_demand.png}
    \caption{Ajout de demande - Placeholder Add Demand}
\end{figure}

La procédure pour saisir une entrée de type "Placeholder - Add demand" est le suivant :

\begin{enumerate}
    
    \item L'utilisateur commence par sélectionner le sub-lever, qui représente la catégorie de la demande d'ajout pour un projet ou bien une croissance stratégique ou autres ensuite il doit cliquer sur le bouton ok.
    \vspace{0.1cm}
    \item Ensuite l'utilisateur doit sélectionner le compte et le contrat qu'il souhaite.
    \vspace{0.1cm}
    \item Puis apparait diffèrent champ qu'il doit remplir, pour l'étape 1 il suffit de sélectionner un OPX-ID qui représente un identificateur d'une opportunité ensuite toutes les informations sont automatiquement remplis.
    \vspace{0.1cm}
    \item La deuxième étape est saisie par l'utilisateur il doit remplir toutes les informations de la demande
    \vspace{0.1cm}
    \item Finalement l'utilisateur a deux choix soit soumettre directement ou bien la soumettre en tant que brouillon s'il n'est pas sur ou souhaite changer les informations de l'entrée dans le futur

\end{enumerate}


\section{Compte - Modification de la masse salariale}

\subsection{Modification de la masse salariale}
\begin{figure}[H]
    \centering
    \includegraphics[scale=0.4,keepaspectratio]{Rapport de stage PFE chez DXC/figures/modification_of_demand.png}
    \caption{Compte - Modification de la masse salariale}
\end{figure}

La modification de demande peut etre soit un "Run-Off" c'est a dire une personne qui quitte son poste ou bien une "extension" ou la personne voit sont contract étendue, le processus est a peu prés le meme, l'utilisateur choisi l'employé ensuite remplie les information necessaire de la deuxieme étape et par la suite saisi son entrés.

\newpage

Dans la plupart des cas l'utilisateur a plusieurs entré a saisire dans ces deux lever, choisir a chaque fois un employé remplire les meme information puis sasire l'entré peut devenir redondant c'est pour cela qu'un module de multi-creation a été fait :

\begin{figure}[H]
    \centering
    \includegraphics[scale=0.4,keepaspectratio]{Rapport de stage PFE chez DXC/figures/multi_modification_of_demand.png}
    \caption{Compte - Modification de la masse salariale en grand nombre}
\end{figure}

La seul difference entre les deux module et le fait que l'utilisateur peut choisir plusieurs utilisateur a la fois ces dernier auront les meme information et le systeme va créer automatiquement une entré par utilisateur selectionné.

\newpage

\subsection{Placeholder Modification de la masse salariale}

De la meme maniere que le placeholder pour l'ajout de demande, le placeholder de modification de demande permet a l'utilisateur de créer des entrée pour des future besoin :

\begin{figure}[H]
    \centering
    \includegraphics[scale=0.4,keepaspectratio]{Rapport de stage PFE chez DXC/figures/placeholder_modification_of_demand.jpg}
    \caption{Placeholder - Modification de la masse salariale}
\end{figure}

Ce dernier ne contient pas tout les informations necessaire pour une entrée run-off mais peut etre ensuite convertit en une entrée VA avec tout les information necessaire.

\newpage
\section{Compte - Consultation des soumissions}

Le module de consultation des soumissions est composé de plusieurs filtre qui vont permettre a l'utilisateur de facilement trouver son entrée 

\begin{figure}[H]
    \centering
    \includegraphics[scale=0.4,keepaspectratio]{Rapport de stage PFE chez DXC/figures/view_submission.jpg}
    \caption{Compte - Consultation des soumissions}
\end{figure}

En plus de cela en cliquant sur une entré nous avons un menu latéral qui vas nous permettre de modifier les information de l'entrer puise enregistre, ou bien 
dupliquer l'entrée mais aussi créer une entrée DCT a partir de cette derniere :

\begin{figure}[H]
    \centering
    \includegraphics[scale=0.4,keepaspectratio]{Rapport de stage PFE chez DXC/figures/account_view_side_menu.png}
    \caption{Compte - Consultation des soumissions - Menu latéral}
\end{figure}

\section{DCT Position Level Management}

\subsection{DCT Position Level Management - Creation de demande}

DCT est l'acronyme de Demand creation template, Pour ce screen l'utilisateur peut créer une demande de croissance : 

\begin{figure}[H]
    \centering
    \includegraphics[scale=0.4,keepaspectratio]{Rapport de stage PFE chez DXC/figures/DCT_Create.png}
    \caption{DCT Position Level Management - Creation de demande }
\end{figure}

Le formulaire est composée de plusieurs champ a remplire avec differentes information, la plupart sont automatiquement remplie a travers les information precedente, ces derniers sont aussi vérifier pour n'avoir que des donnée valide.

\subsection{DCT Position Level Management - Mise a jour de la demande}

Ce screen permet de saisire une modification au niveau d'un DCT, il a le meme principe que le precedent : 

\begin{figure}[H]
    \centering
    \includegraphics[scale=0.4,keepaspectratio]{Rapport de stage PFE chez DXC/figures/DCT_Update.png}
    \caption{DCT Position Level Management - Mise a jour de la demande}
\end{figure}

\subsection{DCT Position Level Management - Consultation des soumissions}

Aprés avoir saisit l'entrée l'utilisateur est envoyé vers le screen de consultation des soumissions:


\begin{figure}[H]
    \centering
    \includegraphics[scale=0.4,keepaspectratio]{Rapport de stage PFE chez DXC/figures/DCT_view_submission.png}
    \caption{DCT Position Level Management - Consultation des soumissions}
\end{figure}

Dans ce dernier nous avont les differents information saisit par l'utilisateur. en cliquant sur l'entrée un menu apparait qui contient quelque bouton, le premier permet de supprimé une entré en cas d'erreur, le deuxiem edit est utilisé pour modifier l'entrée pour corrigé une erreur, il y'a aussi l'option du multi duplicate cette derniere permet a l'utilisateur de dupliquer une entrée en plusieurs si il veut par exemple modifier que quelque information pour chaque entrée et finalement la possibiliter d'attacher un agrément exceptionnel si ce dernier est demander par les personnes qui inspecte l'entrée, d'ailleur en peut voir en haut un commentaire demander a ce dernier d'ajouter le fichier d'agréement.

\subsection{DCT Position Level Management - Validation des entrée}

\begin{figure}[H]
    \centering
    \includegraphics[scale=0.4,keepaspectratio]{Rapport de stage PFE chez DXC/figures/DCT_demand_validation_checks.png}
    \caption{DCT Position Level Management - Validation des entrée}
\end{figure}