\titlespacing*{\chapter}{0pt}{0in}{0.3in}
\chapter*{\makebox[\linewidth]{Introduction}}
\titlespacing*{\chapter}{0pt}{0.45in}{0.3in}%

Du 1er Mars 2022 au 1er septembre 2022 (6 mois), j’ai effectué un stage au sein de l’entreprise \textbf{DXC Technology Maroc} à Rabat. Au cours de ce stage dans le département Business Intelligence, j'ai pu acquérir de nouvelle compétence, de connaitre de nouvelles technologies au-delà de ceux apprise lors de ma formation master et de surtout m'immerger dans le monde professionnel. \\ 

\textbf{DXC Technology Maroc} est une entreprise de technologie dont le siège se trouve dans la technopole de Rabat, Technopolis. Il s’agit d’une coentreprise du groupe \textbf{DXC Technology} et de la \textbf{Caisse de dépôt et de gestion du Maroc}. \textbf{DXC Technology Maroc} a pour vocation d'accompagner les très grands comptes et donneurs d'ordre publics et privés dans leur transformation digitale au niveau de plusieurs service : Data Center, Cloud et Platform, BI & Analytics, Sécurité, Service applicatif, Modern Workplace, Business Process et Conseil.\\ 

Mon stage au département Business Intelligence de \textbf{DXC Technology Maroc} a consisté essentiellement en la conception d'une application de gestion d'effectifs. Tout d'abord on a commencé par l'identification des besoins, ensuite la création de la feuille de route du produit, Puis la création de l'expériences utilisateur avec un design soigné et à partir de là, commencez à développer le backend, après la validation de la qualité de l'application avec une série de test pour finalement déployer l'application, la documenter, et recueillir les commentaires. Plus largement, ce stage a été l’opportunité pour moi d'accéder au monde professionnel. Au-delà du fait d’enrichir mes connaissances en \textbf{Buisness Inteligence}, ce stage m’a permis de comprendre à quelle point la suite \textbf{Microsoft Power Platform} peut faciliter la transformation digitale avec un environnement “Low-Code” voire “No-Code” qui permet de diminuer les différentes étapes de développement d'une application et de production de quelque mois a quelques semaines sans avoir des connaissances approfondies au monde du développement.\\

Pour présenter mon expérience de stage au sein de la société \textbf{DXC Technology Maroc}, nous nous intéresserons a la déscription génerale de l'entreprise (Chapitre 1), puis nous nous pencherons a une etude generale du projet (Chapitre 2) ensuite l'environement technique avec lequels j'ai travaillé durrant ces 6 mois a savoir Microsoft Power Platform (Chapitre 3), Et finnalement la partie de réalisation du projet (Chapitre 4).

\newpage