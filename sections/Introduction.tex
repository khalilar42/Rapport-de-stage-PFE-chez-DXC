\titlespacing*{\chapter}{0pt}{0in}{0.3in}
\chapter*{\makebox[\linewidth]{Introduction Générale}}
\titlespacing*{\chapter}{0pt}{0.45in}{0.3in}%

\vspace{1cm}

Du 1er Mars 2022 au 1er septembre 2022 (6 mois), j’ai effectué un stage au sein de l’entreprise \textbf{DXC Technology Maroc} à Rabat. Au cours de ce stage j'ai pu acquérir de nouvelle compétence, de connaitre de nouvelles technologies au-delà de ceux apprise lors de ma formation master et de surtout m'immerger dans le monde professionnel. \\ 

\textbf{DXC Technology Maroc} est une entreprise dont le siège se trouve dans la technopole de Rabat, Technopolis. Il s’agit d’une coentreprise du groupe \textbf{DXC Technology} et de la \textbf{Caisse de dépôt et de gestion du Maroc}. \textbf{DXC Technology Maroc} a pour vocation d’accompagner les très grands donneurs d’ordre publics et privés dans leur transformation digitale.\\ 

Mon stage au sein de \textbf{DXC Technology Maroc} a consisté essentiellement en la conception et la réalisation d'une application Power Platform pour la gestion de la masse salariale. Tout d'abord on a commencé par l'identification des besoins,puis la création de l'expériences utilisateur avec un design soigné ensuite on a commencez à développer le backend, après cela on est passé à la validation de l'application avec une série de test pour finalement déployer l'application, la documenter, et recueillir les commentaires. Plus largement, ce stage a été l’opportunité pour moi d'accéder au monde professionnel et m’a permis de comprendre à quelle point la suite \textbf{Microsoft Power Platform} peut faciliter la transformation digitale avec un environnement “Low-Code” voire “No-Code” qui permet de diminuer les différentes étapes de développement et de production d'une application de quelque mois a quelques semaines.\\

Pour présenter mon expérience de stage au sein de la société \textbf{DXC Technology Maroc}, nous nous intéresserons à la description générale de l’entreprise et le contexte général du projet (Chapitre 1), puis nous nous pencherons a une étude générale du projet (Chapitre 2), ensuite l’étude des besoins décrivant les contraints et l’étude fonctionnels du projet (Chapitre 3), Par la suite la solution technique, les outils de développement avec lequel j’ai travaillé durant ces 6 mois à savoir Microsoft Power Platform (Chapitre 4), pour finalement présenter la partie de réalisation du projet qui contient les interfaces les plus pertinentes du projet (Chapitre 5).


\newpage