\newpage

\titleformat % design des titres des chapitres
{\chapter}
[display]
{\centering\normalfont\Large\scshape\bfseries}
{\rule[3pt]{0.15\linewidth}{3pt}\quad\chaptertitlename~\thechapter\quad \rule[3pt] {0.15\linewidth}{3pt}}
{0\baselineskip}%espace vertical entre chapitre et nom du chapitre
{\rule{\linewidth}{0.5pt}\break\Huge}
[\vspace{-0.5\baselineskip}\rule{\linewidth}{0.5pt}\vspace{0\baselineskip}]

\let\clearpage\relax% Stop LaTeX from going to a new page; and
\vspace*{5.5cm}%

\chapter{Etudes des besoins}
Le présent chapitre a pour but de définir les besoins fonctionnels et non
fonctionnels, après avoir décrit les processus métiers, et les règles de gestion.

\newpage

\section{Description du projet}
\subsection{Objectifs fonctionnels du projet:}

L’objectif principal du projet est la facilitation de saisie des données ainsi que l’aide
à la prise de décision.
\\

Dans un premier temps, il s’agit de définir de manière claire et précise les besoins
et les attentes d’un système d’information permettant d’automatiser les remontées de
l’équipe vers différentes entités, ainsi que la saisie des Deals, choix des Squads,
l’affectation des ressources et l’affichage des statistiques.



Fournir une visibilité sur les prévisions financières au niveau du compte et du portefeuille, y compris les changements de revenus mensuels et trimestriels