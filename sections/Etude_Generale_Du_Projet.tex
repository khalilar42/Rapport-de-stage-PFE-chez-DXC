\newpage

\titleformat % design des titres des chapitres
{\chapter}
[display]
{\centering\normalfont\Large\scshape\bfseries}
{\rule[3pt]{0.15\linewidth}{3pt}\quad\chaptertitlename~\thechapter\quad \rule[3pt] {0.15\linewidth}{3pt}}
{0\baselineskip}%espace vertical entre chapitre et nom du chapitre
{\rule{\linewidth}{0.5pt}\break\Huge}
[\vspace{-0.5\baselineskip}\rule{\linewidth}{0.5pt}\vspace{0\baselineskip}]

\let\clearpage\relax% Stop LaTeX from going to a new page; and
\vspace*{5.5cm}%

\chapter{ETUDE GENERALE DU PROJET}
Le présent chapitre a pour but de définir les clés du travail
d’identification et planification du projet.

\newpage

\section{Périmètre du projet}
\subsection{Problématique générale}

Actuellement, il n’existe aucune solution pour la gestion d'effectif. Il est
pertinent de libérer le style classique adopté par les commerciaux et les responsables des
ressources humaines qui consiste à utiliser des documents « Excel » pour classifier tous les
acteurs matériaux ou humanitaires qui participent dans la construction d’un « Deal », et
faire des calculs manuels des coûts.
\\

Donc notre problématique majeure, nous devons définir d’une manière claire et précise les
besoins et les attentes d’un système d’information permettant d’automatiser les
remontées de l’équipe vers différentes entités, ainsi que la saisie des Deals, choix des
Squads, l’affectation des ressources et aussi la gestion de la relation client. D’une autre part garder la trace de tous les échanges afin de pouvoir établir des statistiques et prendre la décision du gain du deal.

\subsection{But du projet}
Le but de ce projet est de réaliser une solution power platform pour la facilitation et la gestion de la planification des effectifs. C’est un outil pour la digitalisation du processus de gestion des effectif, ma mission durant ce stage de fin d’études consiste à améliorer cela, les principaus module que j'ai developpé sont:
\\
\begin{itemize}
  \item Accounth Growth form
  \item Accounth Run-Off form
  \item Accounth Migration
  \item Accounth Productivity
  \item Service Line Run-Off
  \item Service Line Migration Form
  \item Service Line Productivity Form
  \item DCT Form
  \item Rebalance Form
  \item Sub Contractor Conversion Form
  \item Sub Contractor Migration Form
  \item Sub Contractor Replacement Form
  \item Sub Contractor Productivity Form
\end{itemize}

\newpage

\subsection{Livrables}
Le tableau suivant reprend les livrables du projet :

\begin{figure}[!h]
    \centering
    \includegraphics[scale=0.55,keepaspectratio]{Rapport de stage PFE chez DXC/figures/livrable.JPG}
    \caption{Les Livrables}
\end{figure}

\subsection{Registre de Gestion des Risques} 

Le “Registre de Gestion des Risques” sera créé et maintenu par le chef de projet
décrivant les risques de toutes natures pouvant affecter la bonne réalisation du projet et
détaillant leur probabilité de réalisation et la sévérité des impacts sur le projet. L’objectif de cette procédure de gestion des risques est d’en maîtriser autant que possible les effets et de permettre la définition et la mise en œuvre de mesures visant à en limiter les effets.


\begin{figure}[!h]
    \centering
    \includegraphics[scale=0.5,keepaspectratio]{Rapport de stage PFE chez DXC/figures/risques.jpg}
    \caption{Registre de Gestion des Risques}
\end{figure}

\subsection{Planification du projet}

Le projet était nouveau pour ma part, Il utilise power platform une technologie non abor-
dées le long de mon parcours scolaire. Du coup, une planification rigoureuse s’est imposée
pour prévoir le déroulement du projet. Grâce aux réunions tenues avec mon encadrant au
sein de DXC, J'ai été éclairés sur les différentes étapes du projet ainsi que leurs
séquencements. 
\\
\\
Le projet est partagé en trois grandes étapes : la première est une phase
de documentation dont les objectifs est de bien assimiler les differents composant de microsoft Power Platform a savoir principalement Power Apps et Power Automate, ensuite la phase de l'elaboration du cahier de charge, vu que le projet avait beacoup de colaborateur cette derniere a pris du temps pour les mettre en accord sur les differents fonctionalitées de l'application, puis une phase de conception de l'application, et vient alors la phase de réalisation de l'application ou j'ai commencer par d'abbord elaborer le front-end pour ensuite passer au back-end et finalement la validation de l'application par les differents membres de l'equipe ainsi que ca documentation.
\\
\\
Le stage a débuté le 1er mars pour une durée de 6 mois. Il en résulte le planning
suivant :
\\
\begin{figure}[!h]
    \centering
    \includegraphics[scale=0.27,keepaspectratio]{Rapport de stage PFE chez DXC/figures/planification_projet_cropped.png}
    \caption{Diagramme de Gantt}
\end{figure}

